\documentclass[uplatex,dvipdfmx]{jsarticle}
\usepackage{amsmath}
\usepackage[dvipdfmx]{graphicx}

\setcounter{tocdepth}{3}
\usepackage{float}
\usepackage{moreverb}
\usepackage{lscape}
\usepackage{url}

%\pagestyle{empty}
%\usepackage{wrapfig}
%\usepackage{url}
%\usepackage{EasyLayout}

\usepackage{ascmac}
%\usepackage{fancybx}

%\pagestyle{myheadings}


\begin{document}


\title{データ表示システムのWebアプリケーション仕様書}
\author{25G1038川治稜雅}
%\date{2015年11月13日}
\maketitle
githubのurlを以下に示す.\\
\url{https://github.com/Runa8931/webpro_06}
\section{利用者向け仕様書}
\subsection{概要}
このシステムはWeb上でデータを閲覧または管理できるアプリケーションである.データの閲覧,詳細確認に加え,データの追加,編集,削除が行える.本システムはうどん,ラーメン,パスタの3つのシステムが閲覧できるが,今回はうどんをもとに説明する.
\subsection{利用できる機能}
本システムで使用可能なものを以下に示す.
\begin{itemize}
    \item{うどんの種類一覧}
    \item{各うどんの詳細データ}
    \item{新規データの登録}
    \item {データの編集}
    \item {データの削除}   
\end{itemize}
\subsection{構成と操作方法について}
\subsubsection{topページ}
Webアプリケーションは以下のurlを使用し,アクセスすることで利用することができる.\\
\url{http://localhost:8080/}
このurlを使用すると最初図\ref{top}のようにtopページに飛ばされる.そこから「うどん一覧」をクリックすることで,「うどんの種類一覧」へ飛ぶことができる.
\begin{figure}[H]\includegraphics[width=12cm]{./top.png}
    \centering
    \caption{topページ}
    \label{top}
    \end{figure}

\subsubsection{一覧表示}
「うどんの種類一覧」へ飛ぶと図\ref{udon}のような画面に移り変わる.ここではうどんの種類一覧を確認することができる.「追加」リンクから新規のデータを追加することができ,「topに戻る」リンクからtopページへ戻ることができる.
\begin{figure}[H]\includegraphics[width=12cm]{./udon.png}
    \centering
    \caption{うどんの種類一覧}
    \label{udon}
    \end{figure}

\subsubsection{詳細表示}
一覧から選択したものの詳細を確認することができる.詳細情報の確認及び編集,削除することができ,図\ref{shousai}のように「編集」リンクを押すと編集画面へ,「削除」リンクを押すと削除画面へ移り変わる.また「うどんの種類一覧に戻る」リンクを押すことでうどんの種類一覧へ戻ることができる.
\begin{figure}[H]\includegraphics[width=12cm]{./shousai.png}
    \centering
    \caption{うどんの種類一覧}
    \label{shousai}
    \end{figure}
\subsubsection{追加機能}
一覧画面から「追加」リンクを押した場合図\ref{tuika}の画面へ移り変わる.
うどんの追加画面では図\ref{tuika2}のように種類名と温度,特徴が入力できる.「送信」ボタンを押すと入力した内容が反映され,図\ref{tuika3}のようになる.
\begin{figure}[H]\includegraphics[width=12cm]{./tuika.png}
    \centering
    \caption{追加画面}
    \label{tuika}
    \end{figure}
\begin{figure}[H]\includegraphics[width=12cm]{./tuika2.png}
    \centering
    \caption{追加中}
    \label{tuika2}
    \end{figure}   
\begin{figure}[H]\includegraphics[width=12cm]{./tuika3.png}
    \centering
    \caption{追加後}
    \label{tuika3}
    \end{figure}     
\subsubsection{編集機能}
詳細画面から「編集」リンクを押した場合図\ref{henshuu}の画面へ移り変わる.
うどんの編集画面では図\ref{hennshuu2}のように種類名と温度,特徴を変更することができる.「送信」ボタンを押すと入力した内容が反映され,図\ref{henshuu3}のようになる.
\begin{figure}[H]\includegraphics[width=12cm]{./henshuu.png}
    \centering
    \caption{編集画面}
    \label{henshuu}
    \end{figure} 
\begin{figure}[H]\includegraphics[width=12cm]{./henshuu2.png}
    \centering
    \caption{編集中}
    \label{henshuu2}
    \end{figure}    
\begin{figure}[H]\includegraphics[width=12cm]{./henshuu3.png}
    \centering
    \caption{編集後}
    \label{henshuu3}
    \end{figure}           
\subsubsection{削除機能}   
詳細画面から「削除」リンクを押した場合図\ref{sakujo}のように削除確認画面が表示される.そこから「削除する」を押すとデータが削除され図\ref{sakujo2}のようになる. 
\begin{figure}[H]\includegraphics[width=12cm]{./sakujo.png}
    \centering
    \caption{削除確認画面}
    \label{sakujo}
    \end{figure}
\begin{figure}[H]\includegraphics[width=12cm]{./udon.png}
    \centering
    \caption{削除後}
    \label{sakujo2}
    \end{figure}
\section{管理者向け仕様書}
\subsection{概要}
    
\end{document}
