\documentclass[uplatex,dvipdfmx]{jsarticle}
\usepackage{amsmath}
\usepackage[dvipdfmx]{graphicx}

\setcounter{tocdepth}{3}
\usepackage{float}
\usepackage{moreverb}
\usepackage{lscape}
\usepackage{url}
%\pagestyle{empty}
%\usepackage{wrapfig}
%\usepackage{url}
%\usepackage{EasyLayout}

\usepackage{ascmac}
%\usepackage{fancybx}

%\pagestyle{myheadings}


\begin{document}


\title{データ表示システムのWebアプリケーション仕様書}
\author{25G1038川治稜雅}
%\date{2015年11月13日}
\maketitle
githubのurlを以下に示す.
\url{https://github.com/Runa8931/webpro_06}
\section{利用者向け仕様書}
\section{概要}
このシステムはWeb上でデータを閲覧または管理できるアプリケーションである.データの閲覧,詳細確認に加え,データの追加,編集,削除が行える.本システムはうどん,ラーメン,パスタの3つのシステムが閲覧できるが,今回はうどんをもとに説明する.
\subsection{利用できる機能}
本システムで使用可能なものを以下に示す.
\begin{itemize}
    \item{うどんの種類一覧}
    \item{各うどんの詳細データ}
    \item{新規データの登録}
    \item {データの編集}
    \item {データの削除}   
\end{itemize}
\subsection{構成と操作方法について}

\end{document}
